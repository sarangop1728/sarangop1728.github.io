% Created 2025-08-12 Tue 13:18
% Intended LaTeX compiler: pdflatex
\documentclass[11pt]{article}
\usepackage[utf8]{inputenc}
\usepackage[T1]{fontenc}
\usepackage{graphicx}
\usepackage{longtable}
\usepackage{wrapfig}
\usepackage{rotating}
\usepackage[normalem]{ulem}
\usepackage{amsmath}
\usepackage{amssymb}
\usepackage{capt-of}
\usepackage{hyperref}
\author{Santiago Arango-Piñeros}
\date{Fall 2025}
\title{Math 411: Introduction to Algebra (Algebra I)}
\hypersetup{
 pdfauthor={Santiago Arango-Piñeros},
 pdftitle={Math 411: Introduction to Algebra (Algebra I)},
 pdfkeywords={},
 pdfsubject={},
 pdfcreator={Emacs 29.3 (Org mode 9.6.15)}, 
 pdflang={English}}
\begin{document}

\maketitle

\section{Course Information}
\label{sec:org8ad96cd}

\begin{center}
\begin{tabular}{lll}
\hline
Section & Time & Location\\[0pt]
\hline
 &  & \\[0pt]
 &  & \\[0pt]
\hline
\end{tabular}
\end{center}

Requests for special arrangements require \textbf{advanced approval} and \textbf{at least two weeks of notice}.  
Contact me if you have further questions, and pay attention to the Make-up Exams Policy!

\section{Course Description}
\label{sec:orgde5896c}
Math 411 is the first part of a two-semester undergraduate abstract algebra
course. The focus of the course will be on studying \emph{groups}. These are algebraic
structures that capture the notion of symmetry. Groups are ubiquitous in all
areas of mathematics (and the world around us). If you commit to this class,
you will master the essential concepts of group theory by the end of the
course.
\subsection{Prerequisites}
\label{sec:org0c47435}
\textbf{MATH 235} and either \textbf{CMPSCI 250} or \textbf{MATH 300}. In other words, we will need some
important concepts from linear algebra, and there will be an emphasis on proofs
and development of careful mathematical reasoning and writing.

\subsection{Main goals}
\label{sec:orgdb5bf81}
\begin{itemize}
\item Survey the basic results and techniques in group theory.
\item Explore key concrete examples and applications.
\item Develop proof skills.
\end{itemize}

There will be three mid-term exams plus a final exam. Regular problem sets will
cover the theoretical part of the course.

My job as your instructor is to provide a framework and guide you in learning
the concepts and methods that comprise the material of the course. Most
learning will take place outside the classroom. Read the textbook carefully and
slowly before the lectures, working through examples and filling in omitted
steps. This is a challenging course with many new concepts and techniques.

\subsection{Learning objectives}
\label{sec:org93c80d8}

\begin{enumerate}
\item To learn the fundamental examples of groups: finitely generated abelian
groups, dihedral groups, symmetric and alternating groups, and matrix
groups.
\item To learn the axiomatic definition of a group and how to use it to prove
basic properties.
\item To learn the concepts of subgroups, cosets, quotients, and how to combine
these to derive Lagrange’s theorem.
\item To learn the concepts of homomorphisms and isomorphisms and Noether's
isomorphism theorems.
\item To understand the structure theorem of finitely generated abelian groups in
terms of the Smith normal form of a matrix with integer coefficients.
\item To learn what it means for a group to act on a set as well as the
natural actions of each of the fundamental examples.
\end{enumerate}

\subsection{Textbook}
\label{sec:orgbe4926d}
\begin{itemize}
\item \textbf{Algebra: Abstract and Concrete} by \emph{Frederick M. Goodman}. The book is
freely available for download at the \href{https://homepage.divms.uiowa.edu/\~goodman/algebrabook.dir/algebrabook.html}{author's web-site}.
\item As a complement of the textbook, we will use some of \href{https://kconrad.math.uconn.edu/blurbs/}{Keith Conrad's blurbs} on
group theory.
\end{itemize}

\section{Contact Information}
\label{sec:org5013f8b}
\begin{itemize}
\item Office: Lederle Graduate Research Tower, Room 1238
\item Email: \href{mailto:sarangopiner@umass.edu}{sarangopiner@umass.edu}
\end{itemize}

\section{Grading Policy}
\label{sec:org9533d7f}
\begin{itemize}
\item Midterm \#1: Thu Oct 5, 7--8:30pm, Room TBA
\item Midterm \#2: Thu Nov 9, 7--8:30pm, Room TBA
\item Final Exam: Fri Dec 15, 10:30am--12:30pm, LGRT 204
\item Homework: 30\% of grade
\begin{itemize}
\item Lowest exam score counts 20\%
\item Other two exams count 25\% each
\end{itemize}
\end{itemize}

\subsection{Grading scale (no rounding)}
\label{sec:orga5ccb5a}

\begin{center}
\begin{tabular}{ll}
\hline
Grade & Range\\[0pt]
\hline
A & [86,100]\\[0pt]
A- & [83,86)\\[0pt]
B+ & [78,83)\\[0pt]
B & [74,78)\\[0pt]
B- & [69,74)\\[0pt]
C+ & [65,69)\\[0pt]
C & [61,65)\\[0pt]
C- & [56,61)\\[0pt]
D+ & [52,56)\\[0pt]
D & [48,52)\\[0pt]
F & [0,48)\\[0pt]
\hline
\end{tabular}
\end{center}

\section{Homework Policy}
\label{sec:org906384d}
\begin{itemize}
\item Due Wednesdays at 10am via Gradescope
\item No late homework
\item Lowest two homework grades dropped
\item Collaboration allowed, but write-ups must be individual
\item Using online solutions (Chegg, ChatGPT, etc.) is cheating
\item Homework is for practice; exams may contain different problems
\end{itemize}

\section{Administrative Details}
\label{sec:org4b5c44e}
\begin{itemize}
\item Add/drop only via SPIRE
\item No waiting list
\item Final exams kept by Math Dept; copies available upon request
\end{itemize}

\section{Class Etiquette}
\label{sec:org353f10d}
\begin{itemize}
\item No texting/calls during lectures except in emergencies
\item Arrive on time; if late, take the nearest seat
\item Laptops/tablets allowed if not disruptive; not allowed during exams
\end{itemize}

\section{Religious Observance}
\label{sec:org5da6afc}
Notify in writing within first two weeks if you will miss class or an exam for religious reasons.

\section{Drops, Withdrawals, and Incompletes}
\label{sec:org9525986}
\begin{itemize}
\item Last day to drop with no record: Mon Sept 11
\item Last day to drop with W: Tue Oct 31
\item Incompletes only for compelling reasons, passing work, and likely completion
\end{itemize}

\section{Make-up Exam Procedure}
\label{sec:orgf72ff3a}
See: \url{http://people.math.umass.edu/\~siman/makeup.html}

\section{Accommodation Statement}
\label{sec:org43c52a0}
UMass Amherst provides equal opportunity for all students.  
Contact Disability Services and notify instructor within first two weeks if accommodations are needed.

\section{Academic Honesty}
\label{sec:org3b888d9}
Academic honesty is required.  
Dishonesty includes cheating, fabrication, plagiarism, and facilitation.  
Sanctions may be imposed.  
Ignorance is not an excuse.
\end{document}
